\documentclass[11pt]{jarticle}
\usepackage[dvipdfmx]{graphicx}
\usepackage{D:\\KCCT\\D5\\jikken\\kcctd-report}
\usepackage{booktabs}
\usepackage{mathcomp}
\usepackage{array}
\usepackage{mathtools,amssymb}
\usepackage{siunitx}
\usepackage{multirow}
\usepackage{tabularx}
\usepackage{subcaption}
\usepackage{float}
\usepackage{listings,jvlisting}
\lstset{
	basicstyle={\ttfamily},
	identifierstyle={\small},
	commentstyle={\smallitshape},
	keywordstyle={\small\bfseries},
	ndkeywordstyle={\small},
	stringstyle={\small\ttfamily},
	frame={tb},
	breaklines=true,
	columns=[l]{fullflexible},
	numbers=left,
	xrightmargin=0zw,
	xleftmargin=3zw,
	numberstyle={\scriptsize},
	stepnumber=1,
	numbersep=1zw,
	lineskip=-0.5ex,
	tabsize=2
}
\renewcommand{\lstlistingname}{ソースコード}

\title{MOS構造の作製と特性評価}
\adviser{西 敬生 教授}

\sdate{令和5年6月15日}
\edate{令和5年6月日}
\fdate{令和5年7月日}
\rdate{}

\grade{5}
\anumber{12}
\gnumber{B}
\name{河合 将暉}
\jname{}
\comment{}
\begin{document}
\maketitle

\section{目的}
	トランジスタ製作の基本技術の習得とMOSトランジスタの基本特性であるMOS容量の電圧依存性、周波数特性および酸化膜厚の参加時間依存性について学ぶ。

\section{解説}
	\subsection{MOS構造}
		MOSとはMetal(金属)\-Oxied(酸化膜)\-Semiconductor(半導体)の頭文字の略称である。
		半導体Siの表面を酸化させ、絶縁体酸化膜\,SiO_{2}が形成される。
		この上に金属電極を積むことで図\refeq{}のMOS構造が形成される。

	\subsection{作製過程}
		\begin{enumerate}
			\item ウェーハ洗浄
				Siウェーハの表面は一度パッケージから出してしまえば、たとえクリーンルーム内といえども、多かれ少なかれ汚染される。
				本校クリーンルームはクラス10000(1立方フィート内に$0.5\,\mu$\,mの粒子が1万粒)とされ、専門家ではない萼せいがアツカウことを考えれば、ウェーハを扱う企業の現場(クラス1〜100)より非常に汚染されやすい環境にある。
				具体的にウェーハ表面を汚染するものや除去したいものとしては

				\begin{enumerate}
					\item パーティクル\\
					\item アルカリ金属、重金属\\
					\item 有機物\\
					\item Si自然酸化膜\\
				\end{enumerate}

				以上が挙げられる。
				ここでのパーティクルとは、材質などは問わずに、粒形が数百nm以上のものの総称である。

				これらをウェーハ表面に物理的・科学敵にダメージを与えることなく、除去する洗浄方法が必要とされており、RCA法など、多くの方法が提案されている。

			\item 酸化膜形成
				Siの酸化膜はウェーハを、酸素を満たした900〜1200^\circ の高温の炉中に入れr熱酸化によって形成されることが多い。
				満たす酸素の供給源としては、乾燥した純粋な酸素O_{2}を送り込むドライ酸化と、水蒸気または水素と酸素の混合気を送り込むウェット(水蒸気またはスチームなどともいう)酸化がある。

				酸化のメカニズムは、
					\begin{enumerate}
						\item 酸化種(O_{2}またはH_{2}O)が表面で反応もしくはSiO_{2}に吸着される。\\
						\item 吸着されたO_{2}またはH_{2}Oが酸化膜SiO_{2}の中を拡散してシリコンの界面に達する。\\
						\item シリコンとの界面でシリコンと反応してSio_{2}になる。\\
					\end{enumerate}
				といった段階を経る。
				酸化速度は酸化膜SiO_{2}が薄い時には③の化学反応の速度で決まり、厚い時には②の拡散する速度によって決まる。
				前者の状況を反応律速、後者を供給律速という。

				全体の反応を式で表すと
				\begin{eqation}
					T^{2}_{OX} + AT_{OX} = B(t+\tau_{0})
				\end{eqation}
				となり、ここでA,Bは温度と酸化条件で決まる定数、

			\item フォトリソグラフィ
		\end{enumerate}

	\subsection{MOS構造の電気的特性}
		

\section{実験方法}
	\subsection{使用器具}
	\subsection{実験方法}
\section{実験結果}
\section{考察}
\section{感想}
\begin{thebibliography}{99}
\bibitem{ref:指導書}
「実験実習指導書」神戸高専電子工学科 pp,
\end{thebibliography}
\end{document}
