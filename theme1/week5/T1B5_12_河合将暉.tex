\documentclass{ltjsarticle}
\usepackage{booktabs}
\usepackage{mathcomp}
\usepackage{array}
\usepackage{mathtools,amssymb}
\usepackage{siunitx}
\usepackage{multirow}
\usepackage{tabularx}
\usepackage{subcaption}
\usepackage{float}
\usepackage{kcctd-report}
\usepackage{listings,jvlisting}

\title{光情報通信に関する実験}
\adviser{荻原昭文}

\sdate{令和5年12月14日}
\edate{令和5年12月21日}
\fdate{令和5年12月26日}
\rdate{}

\grade{5}
\anumber{}
\gnumber{B}
\name{河合将暉}
\jname{岡田}
\comment{}
\begin{document}
\maketitle

\section{目的}
	LED,レーザ等による発光現象の測定やフォトダイオードによる光検出回路の作製実験
	を通じて光情報通信の原理を理解する.

\section{解説}
	\subsection{圧電効果と逆圧電効果}
	\subsection{圧電セラミック素子}
\section{実験方法}
	\subsection{使用器具}
		\begin{table}[H]
		\begin{center}
		\caption{使用器具}
		\begin{tabular}{clllll} \toprule
		No&\multicolumn{1}{c}{機器名}&\multicolumn{1}{c}{企業名}&\multicolumn{1}{c}{型番}&\multicolumn{1}{c}{シリアルNo}&\multicolumn{1}{c}{備考}\\
		\hline
		1&DMM&&&\\
		2&オシロスコープ&&&\\
		3&FG&&&\\
		4&オペアンプ&&&\\
		5&圧電セラミック素子&&&\\
		6&レーザダイオード&&&\\
		7&2分割フォトダイオード&&&\\
		8&各種回路素子&&&\\
		\bottomrule
		\end{tabular}
		\end{center}
		\end{table}

	\subsection{圧電セラミック素子への電圧印加による共振現象の測定}
		\subsubsection{実験1}
			圧電セラミック素子は,形状の異なる三種類(a:大,b:中,c:小)を用いるが,
			はじめに,圧電セラミック素子aを用いて行った.図\refeq{}に示すように,
			測定システムを構成し,圧電セラミック素子の端子にファンクションジェネレータ
			からの出力を接続した.また,電圧と周波数計測用にデジタルマルチメータの端子
			も接続する.正弦波信号を選択し,周波数は2.0\,kHz~4.0\,kHzの範囲で
			0.1\,kHz程度ごとに変化させた.この時の印加電圧値はデジタルマルチメータ
			によって測定し,3.0\,Vに設定した.次項の実験2,実験3でも同様に3.0\,Vを印加した.

			レーザダイオード(LD)から射出したレーザビームを調整用のミラーの角度を変えながら,
			圧電セラミック素子の金属プレート表面に入射させ,この反射光を2分割フォトダイオード
			のAとBの2つの受光部の中央あたりに入射するように調整ミラー上の調整ネジを回して設定した.
			差分増幅回路のゲイン抵抗($\mathrm{R_G}$)は信号の検出がしやすいようにゲインが大きくなる
			$100\,\Omega 200\,\Omega$ 程度の抵抗値の抵抗値にした.実験1~実験3において,同じゲイン抵抗値を用いた.
			また,出力信号の振幅が最も大きく検出された周波数の波形を記録し,この時の周波数と振幅値の値を測定した.

		\subsubsection{実験2}
			次に,圧電セラミック素子bに素子を変更して実験を行ったが,正弦波信号の周波数は
			4.0\,kHz~6.0\,kHzの範囲で0.1\,kHz程度ごとに変化させた.この時の印加電圧値は,
			3.0\,Vにし,ゲイン抵抗は実験1と同様の値を用いた.また,出力信号の振幅が最も大きく検出された周波数の波形を記録し,
			この時の周波数と振幅値の値を測定した.

		\subsubsection{実験3}
			最後に,圧電セラミック素子cに素子を変更して同様の実験を行ったが,周波数は
			6.0\,kHz~9.0\,kHzの範囲で0.1\,kHz程度ごとに変化させた.この時の印加電圧値は,
			3.0\,Vにし,ゲイン抵抗は実験1と同様の値を用いた.また,出力信号の振幅が最も大きく検出された周波数の波形を記録し,
			この時の周波数と振幅値の値を測定した.

	\subsection{音源の周波数変化による振動状態の光測定}
		\subsubsection{差分増幅回路を用いた光信号検出}
			\subsubsection{実験1:ゲイン抵抗($\mathrm{R_G}$)を変化した場合の出力信号の測定実験}
				
				図に示すような差分増幅回路と光学系を構成し,半導体レーザから出射した
				レーザビームを調整用ミラーの角度を変えながら,音源が入るプラスチックBOX(大)の入口に
				取り付けたミラーに照射した.このミラーから2分割フォトダイオードまでの距離は,15~30\,cmに設定した.
				このミラーから反射されたレーザビームを2分割フォトダイオードのAとBの2つの受光部の中心あたりに入射させた.
				出力波形が明瞭に観察できる光照射位置に調整した.

				差分増幅回路のゲイン抵抗($\mathrm{R_G}$)を調整し,オシロスコープ上に出力波形をモニターした.
				この時,差分増幅回路のゲイン抵抗値をいくつかの値に変化し,音源の周波数設定を変更しながら出力信号波形を
				検出しデータを取得した.

				音源としては,遠隔で操作できるBluetooth対応の小型スピーカーをプラスチック容器に封入し,
				入口にレーザ反射用のミラーを取り付ける.小型スピーカへの信号入力はPC上にインストールされたフリーソフトの
				WabeGeneにより正弦波(sin波)を入力した.出力(音の大きさ)は,PCの音声出力設定で行った.
			\subsubsection{実験2:材料の違いによる出力信号測定}
			\subsubsection{実験3:音源から検出器までの距離を変化させた場合の出力信号測定}
\section{実験結果}
\section{考察}
\begin{thebibliography}{99}
\bibitem{指導書}

\end{thebibliography}
\end{document}