\documentclass[11pt]{jarticle}
\usepackage[dvipdfmx]{graphicx}
\usepackage{kcctd-report}
\usepackage{booktabs}
\usepackage{mathcomp}
\usepackage{array}
\usepackage{mathtools,amssymb}
\usepackage{siunitx}
\usepackage{multirow}
\usepackage{tabularx}
\usepackage{subcaption}
\usepackage{float}
\usepackage{listings,jvlisting}
\lstset{
	basicstyle={\ttfamily},
	identifierstyle={\small},
	commentstyle={\smallitshape},
	keywordstyle={\small\bfseries},
	ndkeywordstyle={\small},
	stringstyle={\small\ttfamily},
	frame={tb},
	breaklines=true,
	columns=[l]{fullflexible},
	numbers=left,
	xrightmargin=0zw,
	xleftmargin=3zw,
	numberstyle={\scriptsize},
	stepnumber=1,
	numbersep=1zw,
	lineskip=-0.5ex,
	tabsize=2
}
\renewcommand{\lstlistingname}{ソースコード}

\title{}
\adviser{}

\sdate{令和5年6月29日}
\edate{令和5年7月日}
\fdate{令和5年7月日}
\rdate{西 敬生 教授}

\grade{5}
\anumber{12}
\gnumber{B}
\name{河合 将暉}
\jname{
		岡田 あきたか 川邊 愛貴 きゅうと 久米 りょうと
	  }
\comment{}
\begin{document}
\maketitle

\section{目的}
	半導体素子を作成する上で最重要技術である不純物拡散によるシリコンPN接合作成技術を習得し、半導体素子や集積回路の作成技術に関する基本概念を得ることを目的とする
\section{解説}
	\subsection{拡散}
	\subsection{半導体}
	\subsection{伝導型}
	\subsection{PN接合}
	\subsection{ショットキー接触}

\section{実験内容}
	\subsection{使用器具}
		\begin{table}[H]
		\begin{center}
		\caption{使用器具}
		\label{tab:used}
		\begin{tabular}{clllll} \toprule
		No&\multicolumn{1}{l}{機器名}&\multicolumn{1}{l}{型番}&\multicolumn{1}{l}{シリアルNo}&\multicolumn{1}{l}{備考}\\ \hline
		1&電気炉&&&\\
		2&ディップコーター&&&\\
		3&真空蒸着装置&&&\\
		4&ホットプレート&&&\\
		5&ジェットオーブン&&&\\
		6&定電圧電源&&&\\
		7&ディジタルマルチメータ&&&\\ \bottomrule
		\end{tabular}
		\end{center}
		\end{table}

	\subsection{実験方法}
		\subsubsection{pn接合試料作成手順}
		\subsubsection{測定方法}
			\begin{enumerate}
				\item pn接合ダイオードの順方向・逆方向I−V特性\\
				\item 作成したpn接合の順方向・逆方向I−V特性\\
			\end{enumerate}

\section{実験結果}
	\subsection{pn接合ダイオードの順方向・逆方向I−V特性}
	\subsection{作成したpn接合の順方向・逆方向I−V特性}

\section{考察}
	\subsection{pn判定}
	\subsection{I−V特性}
	\subsection{片対数の結果}
\section{感想}
\begin{thebibliography}{99}
\bibitem{ref:指導書}
西 敬生「実験実習指導書」神戸高専電子工学科 pp.15-16
\end{thebibliography}
\end{document}