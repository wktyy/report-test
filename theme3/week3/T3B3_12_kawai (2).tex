\documentclass[11pt]{jarticle}
\usepackage[dvipdfmx]{graphicx}
\usepackage{kcctd-report}
\usepackage{booktabs}
\usepackage{mathcomp}
\usepackage{array}
\usepackage{mathtools,amssymb}
\usepackage{siunitx}
\usepackage{multirow}
\usepackage{tabularx}
\usepackage{subcaption}
\usepackage{float}
\usepackage{listings,jvlisting}
\lstset{
	basicstyle={\ttfamily},
	identifierstyle={\small},
	commentstyle={\smallitshape},
	keywordstyle={\small\bfseries},
	ndkeywordstyle={\small},
	stringstyle={\small\ttfamily},
	frame={tb},
	breaklines=true,
	columns=[l]{fullflexible},
	numbers=left,
	xrightmargin=0zw,
	xleftmargin=3zw,
	numberstyle={\scriptsize},
	stepnumber=1,
	numbersep=1zw,
	lineskip=-0.5ex,
	tabsize=2
}
\renewcommand{\lstlistingname}{ソースコード}

\title{MOS構造の作製と特性評価}
\adviser{西 敬生 教授}

\sdate{令和5年6月15日}
\edate{令和5年6月22日}
\fdate{令和5年6月27日}
\rdate{}

\grade{5}
\anumber{12}
\gnumber{B}
\name{河合 将暉}
\jname{}
\comment{}
\begin{document}
\maketitle

\section{目的}
	トランジスタ製作の基本技術の習得とMOSトランジスタの基本特性であるMOS容量の電圧依存性,周波数特性および酸化膜厚の酸化時間依存性について学ぶ.

\section{解説}
	\subsection{MOS構造}
		MOSとはMetal(金属)−Oxied(酸化膜)−Semiconductor(半導体)の頭文字の略称である.
		半導体Siの表面を酸化させ,絶縁体酸化膜\,$\mathrm{SiO_{2}}$が形成される.
		この上に金属電極を積むことで図\refeq{fig:MOSstructure}のMOS構造が形成される.

		\begin{figure}[H]
		\centering
		\includegraphics[width = 10cm]{figs/IMG_0209.JPG}
		\caption{MOS構造図}
		\label{fig:MOSstructure}
		\end{figure}

	\subsection{作製過程}
		\begin{enumerate}
			\item ウェーハ洗浄\\
				Siウェーハの表面は一度パッケージから出してしまえば,たとえクリーンルーム内といえども,多かれ少なかれ汚染される.
				本校クリーンルームはクラス10000(1立方フィート内に$0.5\,\mu$\,mの粒子が1万粒)とされ,専門家ではない学生が扱うことを考えれば,ウェーハを扱う企業の現場(クラス1〜100)より非常に汚染されやすい環境にある.
				具体的にウェーハ表面を汚染するものや除去したいものとしては

				\begin{enumerate}
					\item パーティクル\\
					\item アルカリ金属,重金属\\
					\item 有機物\\
					\item Si自然酸化膜\\
				\end{enumerate}

				以上が挙げられる.
				ここでのパーティクルとは,材質などは問わずに,粒形が数百nm以上のものの総称である.

				これらをウェーハ表面に物理的・科学敵にダメージを与えることなく,除去する洗浄方法が必要とされており,RCA法など,多くの方法が提案されている.

			\item 酸化膜形成\\
				Siの酸化膜はウェーハを,酸素を満たした$900〜1200^\circ$ の高温の炉中に入れ熱酸化によって形成されることが多い.
				満たす酸素の供給源としては,乾燥した純粋な酸素$\mathrm{O_{2}}$を送り込むドライ酸化と,水蒸気または水素と酸素の混合気を送り込むウェット(水蒸気またはスチームなどともいう)酸化がある.

				酸化のメカニズムは,

				\begin{enumerate}
					\item 酸化種($\mathrm{O_{2}}$または$\mathrm{H_{2}O}$)が表面で反応もしくは$\mathrm{SiO_{2}}$に吸着される.\\
					\item 吸着された$\mathrm{O}_{2}$または$\mathrm{H_{2}O}$が酸化膜$\mathrm{SiO_{2}}$の中を拡散してシリコンの界面に達する.\\
					\item シリコンとの界面でシリコンと反応して$\mathrm{SiO_{2}}$になる.\\
				\end{enumerate}
				
				といった段階を経る.
				酸化速度は酸化膜$\mathrm{SiO_{2}}$が薄い時には3の化学反応の速度で決まり,厚い時には2の拡散する速度によって決まる.
				前者の状況を反応律速,後者を供給律速という.

				全体の反応を式で表すと
				\begin{equation}
					T^{2}_{OX} + AT_{OX} = B(t+\tau_{0})
				\end{equation}

				となり,ここでA,Bは温度と酸化条件で決まる定数,$\tau_{0}$は初期の酸化膜厚に対応する定数である.
				酸化時間tが長くて,$T_{OX}$が厚いときには

				\begin{equation}
					T_{OX}^{2}\backsimeq (B/A)(t+\tau_{0})
				\end{equation}

				となる.
				これらの酸化定数を表\refeq{tab:Sidrai}に示す.

				\begin{table}[H]
				\begin{center}
				\caption{シリコンのドライ酸化時の酸化定数}
				\label{tab:Sidrai}
				\begin{tabular}{cSSS} \toprule
					酸化温度T\,[$^\circ \mathrm{C}$]&A[$\mathrm{\mu m}$]&B[$\mathrm{\mu m^{2}/h}$]&$\tau_{0}$[h]\\ \hline
					1200&0.040&0.045&0.027\\
					1100&0.090&0.027&0.076\\
					1000&0.165&0.0117&0.37\\
					920&0.235&0.0049&1.40\\
					800&0.370&0.0011&9.0\\ \bottomrule
				\end{tabular}
				\end{center}
				\end{table}

			\item フォトリソグラフィ
				IC製造において各種材料膜を所望の形状加工にするため,パターニングを施すことをフォトリソグラフィと呼ぶ.
				この工程は通常の写真技術の応用であり,以下の手順を踏む.
				\begin{enumerate}
					\item レジスト塗布(感光剤塗布):ウェーハ上にフォトレジストという感光性樹脂をコーティング
					\item 露光:平面的パターンが書かれたマスクを通し,光を照射して,パターンをレジストに転写
					\item レジスト部分除去:レジストの化学反応により変質した部分を除去することでマスクパターンと同じレジストパターンが形成される.
					\item 加工:部分的なレジストの除去により,レジストの下の層が一部分,表出.その表出部分をエッチングしたり,上から蒸着や塗布などを行ったりすることで,パターンと同じ構造を形成できる.
				\end{enumerate}

				$\boldsymbol{レジスト}$\\
					レジストには光が当たった部分が残るネガ型と,光が当たった部分が解けて取れるポジ型があり,本実験ではポジ型のOFPRという商品名で,光が当たるとアルカリ溶液に可溶性となる.
					レジストの塗布にはスピナー(スピンコーター)と呼ばれる塗布機を用いる.
					これはウェーハにレジストを滴下後高速回転させることで一定の膜厚のレジストのコーティングを可能とする.\\
				\\
				$\boldsymbol{露光}$\\
					マスクアライナと呼ばれる.マスク接触型の露光装置で行われる.光学ステージへのセットは自動で行われる.
					露光の高原には高圧水銀ランプから発せられる紫外線(i線,波長$\lambda = 365$nm)


			\item フォトリソグラフィ\\
				IC製造において各材料膜を所望の形状に加工するため,パターニングを施すことをフォトリソグラフィと呼ぶ.
				この工程は通常の写真技術の応用であり,以下の手順を踏む.
				\begin{enumerate}
					\item レジスト塗布(感光剤塗布)\\
						ウェーハ上にフォトレジストという感光性樹脂をコーティング
					\item 露光\\
						平均的パターンが描かれたマスクを通し,光を照射して,パターンをレジストに転写
					\item レジスト部分除去\\
						レジストの光化学反応により変質した部分を除去することで,マスクパターンと同じレジストパターンが形成される.
					\item 加工\\
						部分的なレジストの除去により,レジストの下の層が一部分表出する.
						その表出部分をエッチングしたり,上から蒸着や塗布を行うことで,パターンと同じ構造を形成できる.
				\end{enumerate}
		\end{enumerate}

	\subsection{MOS構造の電気的特性}
		\begin{enumerate}
			\item 電圧印加による3つの状態\\
				MOS構造は,図\refeq{fig:kubouMOS}のように印加電圧によって電気的に3つの状態に変化する.
				\begin{figure}[H]
				\centering
				\includegraphics[width = 8cm]{figs/IMG_0210.JPG}
				\caption{空乏状態のMOS構造}
				\label{fig:kubouMOS}
				\end{figure}
				図\refeq{fig:kubouMOS}はプラスの電圧V1を印加した状態で,酸化膜を挟んで電極下の半導体の部分に電子が引き寄せられ,多数キャリアの正孔と引き寄せられた電子が再結合し,キャリアの存在しない空乏層と呼ばれる領域が生じる.
				これを空乏状態という.
				V1より大きな電圧を印加するとより多く電子が引き寄せられ,酸化膜−半導体界面のごく薄い部分だけP形からN形に``反転''する.
				これを図\refeq{fig:hantenMOS}のように反転層といい,反転状態と呼ぶ.
				\begin{figure}[H]
				\centering
				\includegraphics[width = 8cm]{figs/IMG_0211.JPG}
				\caption{反転状態のMOS構造}
				\label{fig:hantenMOS}
				\end{figure}
				このようにMOS構造では,半導体の伝導型を電圧によって変化させることができる.
				MOSトランジスタはこれを利用している.
				図\refeq{fig:tikusekiMOS}は蓄積状態と呼ばれ,‐V3を印加すると正孔を引き寄せ,その部分の抵抗率が下げられる.
				\begin{figure}[H]
				\centering
				\includegraphics[width = 8cm]{figs/IMG_0212.JPG}
				\caption{蓄積状態のMOS構造}
				\label{fig:tikusekiMOS}
				\end{figure}

			\item 電圧−容量特性\\
				MOS構造は金属で絶縁体をサンドイッチしていると考えるとコンデンサの一種と見ることもできる.
				空乏層という絶縁層を生じさせ,印加電圧によって空乏層幅を変化させられるため,静電容量も変えることができる.
				図\refeq{fig:MOSC-V}はMOS構造のC−V特性で周波数によって特性が図のように変化する.
				\begin{figure}[H]
				\centering
				\includegraphics[width = 8cm]{figs/IMG_0213.JPG}
				\caption{MOS構造のC‐V特性}
				\label{fig:MOSC-V}
				\end{figure}

		\end{enumerate}

\section{実験方法}
	\subsection{使用器具}
		本実験での使用器具を表\refeq{tab:used}に示す.
		\begin{table}[H]
		\begin{center}
		\caption{使用器具}
		\label{tab:used}
		\begin{tabular}{clllll} \toprule
		No&\multicolumn{1}{l}{機器名}&\multicolumn{1}{l}{型番}&\multicolumn{1}{l}{シリアルNo}&\multicolumn{1}{l}{備考}\\ \hline
		1&横型管状電気炉&&&\\
		2&ドラフター&&&\\
		3&スピンコーター&&&\\
		4&ジェットオーブン&&&\\
		5&真空蒸着装置&&&\\
		6&超音波洗浄機&&&\\
		7&ホットプレート&&&\\
		8&マスクアライナ&&&\\
		9&顕微鏡&&&\\
		10&LCRメータ&&&\\
		11&定電圧電源&&&\\
		12&グローブボックス&&&\\
		13&シリコン基板&&&\\ \bottomrule
		\end{tabular}
		\end{center}
		\end{table}

	\subsection{実験方法}
		\cite{ref:指導書}より,MOS構造の作成工程を以下に示す.
		完成したMOS構造の静電容量−電圧特性,周波数依存性を測定した.
		\begin{enumerate}
			\item プライムウェーハ初期洗浄\\
				排気装置であるドラフターの中で表\refeq{tab:wash}の手順に従って洗浄作業を行った.
				\begin{table}[H]
				\begin{center}
				\caption{初期洗浄方法}
				\label{tab:wash}
				\begin{tabular}{c|ccc} \toprule
					手順&洗浄方法&時間\,[分]&使用ビーカ\\ \hline
					1&セミコクリーン23による超音波洗浄&5&セミコクリーン\\
					2&純水オーバーフロー&5&セミコクリーン−$\mathrm{H_{2}O}$\\
					3&遠心乾燥&1&\\ \bottomrule
				\end{tabular}
				\end{center}
				\end{table}

			\item 酸化膜形成\\
				酸化膜形成の条件を以下に示す.
				\begin{itemize}
					\item 電気炉の温度:$1060^\circ \mathrm{C}$
					\item レギュレータ圧力:酸素 3\,L/min,窒素 3\,L/min
					\item ガス流量:酸素 1\,kg/$\mathrm{cm^{2}}$,窒素 1.2\,kg/$\mathrm{cm^{2}}$
					\item 酸化時間:50分,65分,80分
				\end{itemize}
				実際には電気炉を稼働させてから最初の20分は温度が安定せず,一時的に$1160^\circ \mathrm{C}$まで上昇した.

				次に,酸化膜形成の手順を以下に示す.
				\begin{enumerate}
					\item 窒素を流すスイッチ(1−SV)がONで,窒素フローメーターが3\,L/minであることを確認した.
					\item ウェーハを石英製ホルダーにセットし,そのホルダーごと電気炉の入口の手前(電気炉の外)に置いた.
					\item 手袋をし,電気炉の入口に石英棒で,ウェーハをゆっくりと中央に挿入し,蓋をした.
					\item 酸素を流すスイッチ(2‐SV)をON,酸素フローメータが3\,L/minであることを確認した.
					\item 1−SVをOFF,窒素を0にした.
					\item 酸化時間後,1−SVをON,窒素フローメータ3\,L/minであることを確認した.
					\item 2−SVをOFF,酸素を0にした.
					\item 手袋をし,電気炉の蓋を取った.
					\item ウェーハをゆっくりと取り出し,入口のところで2分間置いた.
					\item 電気炉の外まで引出し,3分置いた後,トレイからウェーハを1枚取った.
					\item (c)に戻り,全てのウェーハを所定時間酸化し終わるまで続けた.
				\end{enumerate}

			\item 表面Al蒸着\\
				ゲートとなる金属(Al)をウェーハ表面に蒸着した.
				蒸着減となるタングステンヒータとAl塊を表\refeq{tab:tangsten}のように有機洗浄した.
				その後,真空蒸着装置にウェーハ,タングステンヒータとAl塊をセットし,装置内を真空にした後,蒸着.
				\begin{table}[H]
				\begin{center}
				\caption{タングステンヒータとアルミの有機洗浄方法}
				\label{tab:tangsten}
				\begin{tabular}{c|ccc} \toprule
					手順&洗浄方法&時間\,[分]&使用ビーカ\\ \hline
					1&アセトン&5&AL‐US\\
					2&メタノール&5&AL‐US\\
					3&アセトン&5&AL‐US\\
					4&メタノール&5&AL‐US\\
					5&ホットプレート上で乾燥&&\\ \bottomrule
				\end{tabular}
				\end{center}
				\end{table}
			\item フォトリソグラフィ\\
				蒸着したAl薄膜を2mm角の電極パターンとして形成するため,フォトリソグラフィを行った.
				使ったレジストは感光部が溶けるポジ型を使用するため,定着液も必要となった.
				窒素中プリべークのあと,コンタクトアライナ方式により露光され,現像後,ポストベークを行った.
				フォトリソグラフィの手順は以下に示す.
				\begin{enumerate}
					\item ジェットオーブンの設定温度が110度,窒素流量が5\,L/minであることを確認した.
					\item ウェーハを金属網の上に載せ,網ごとジェットオーブンに入れ,3分間ベーキングを行った.
					\item ジェットオーブンから網ごとウェーハを取り出した.
					\item スピナーの電源が入っていること,壁にある真空弁が開いていることを確認した.
					\item スピナーの中央部にある試料台にAl電極面を上に向けてウェーハを置いた.
					\item 真空チャックスイッチをONにした.
					\item 定着液(OAP)をスポイトで滴下後,スピナーをスタートさせた.
					\item ポジレジスト(OFPR‐800)をスポイトで滴下した.レジストの粘度が高いため,直径2.5cm程度の円になるように多めに滴下する.
					\item ウェーハを網の上に置き,残りのウェーハも同様の手順を行った.
					\item ジェットオーブンに網ごとウェーハを入れ,90秒ベーキングを行った.
					\item 網ごとウェーハをマスクアライナのところに運んだ.
					\item マスクアライナのマスクホルダ両脇にあるマスクロックレバーを外側に向け,リリースした.
					\item マスクフォルダを持ち上げて開き,試料台の中心が合い,円の中に試料が入るようにウェーハを設置した.
					\item 試料固定ボタンをONにした.
					\item マスクフォルダを閉じ,両脇のマスクロックレバーを内側にしてロックした.
					\item マスクパターンがウェーハのAlの中に完全に収まるようにX,Yまわしで調整した.
					\item 試料台Z軸動作ボタンをONにして,8秒間露光した.
					\item ランプハウスを後ろに移動させ,試料台Z軸動作ボタンをOFFにした.
					\item マスクホルダの両脇のマスクロックレバーを外側にして,リリースし,マスクホルダを開いた.
					\item 試料固定スイッチをOFFにし,試料を取り出した.残りのウェーハについても同様に行った.
					\item 露光を終えたウェーハをフッ素樹脂製ディッパーに装着した.
					\item 2台のスターラーの電源が入っていることを確認した.
					\item ウェーハを1枚ずつ,撹拌中の現像液の中に2分間ディッパーごと浸した.感光した部分だけのレジストが落ち,マスクの模様が反転出現することを確認した.
					\item ウェーハ上の水分を飛ばすため,窒素ガンでウェーハに対して窒素を吹きかけた.
					\item 窒素流量5\,L/minのジェットオーブンにウェーハを入れ,$110\,^\circ \mathrm{C}$ で3分間ベーキングを行った.
				\end{enumerate}

			\item Alエッチング\\
				パターン通りにAlエッチングし,電極を形成する.
				エッチングの条件は表\refeq{tab:aletching}に示す.
				\begin{table}[H]
				\begin{center}
				\caption{アルミエッチングの方法}
				\label{tab:aletching}
				\begin{tabular}{c|cccc} \toprule
					手順&洗浄方法&温度\,[$^\circ \mathrm{C}$]&時間\,[分]&使用ビーカ\\ \hline
					1&エッチング液 泡が出なくなるまで浸す&43&泡が出なくなるまで&AL‐ETCH\\
					2&純水オーバーフロー&&1&AL‐$\mathrm{H_{2}O}$\\
					3&遠心乾燥&&1&\\ \bottomrule
				\end{tabular}
				\end{center}
				\end{table}
				エッチング液はホットプレートで43度に上げた.
				温度はエッチング液に入れた温度計で測定した.
				43度のエッチング液は高い粘性を示した.
				ウェーハを単独で入れてしまうと後で取り出しにくいので,ウェーハはピンセットに挟んだまま液に浸した.
				Alの溶液はエッチング液中に気泡が生じることによって確認できた.
				泡がなくなったとき,エッチングを終了し,純水オーバーフローを行った.

			\item レジスト除去\\
				Al電極上のレジストを有機溶媒で除去.
				ウェーアトレーごと洗浄し,メタノールに浸すときはホットプレートを使用した.
				表\refeq{tab:resistwash}にレジスト除去の方法を示す.
				\begin{table}[H]
				\begin{center}
				\caption{レジスト除去の方法}
				\label{tab:resistwash}
				\begin{tabular}{c|cccc} \toprule
					手順&洗浄方法&温度\,[$^\circ \mathrm{C}$]&時間\,[分]&使用ビーカ\\ \hline
					1&リンス1&&3&AL‐RINSE1\\
					2&リンス2&&1&AL‐RINSE1\\
					3&アセトン&&浸す&AL‐ACETONE\\
					4&メタノール&沸騰&浸す&AL‐METHANOL\\
					5&ホットプレートの紙上で乾燥&150&&\\ \bottomrule
				\end{tabular}
				\end{center}
				\end{table}

			\item 表面レジスト塗布\\
				この後の行程の裏面酸化膜エッチングでの表面酸化膜および,Al電極の保護のため,表面をレジストでコーティングした.
				手順はフォトリソグラフィの(a)から(j)までをもう一度行った.

			\item 裏面酸化膜エッチング\\
				エッチングにはバッファードフッ酸を用いた.
				シリコンはバッファードフッ酸に対して撥水性を有するためウェーハがエッチング液をはじくのを確認したら,その後30秒間フッ酸に浸し,純水オーバーフローを行った.
				以下に手順を表\refeq{tab:uraetching}に示す.
				\begin{table}[H]
				\begin{center}
				\caption{裏面酸化膜エッチングの方法}
				\label{tab:uraetching}
				\begin{tabular}{c|ccc} \toprule
					手順&洗浄方法&時間\,[分]&使用ビーカ\\ \hline
					1&バッファードフッ酸&ウェーハがフッ酸をはじくまで +0.5&G‐BHF\\
					2&純水オーバーフロー&5&G‐$\mathrm{H_{2}O}‐BHF$\\
					3&遠心乾燥&1&\\ \bottomrule
				\end{tabular}
				\end{center}
				\end{table}

			\item レジスト除去\\
				表\refeq{tab:resistwash}と同様の手順で表面のレジストを除去した.

			\item Al蒸着\\
				タングステンヒータとアルミ塊の洗浄を行い表面と同様にアルミ蒸着を行った.
				ウェーハは裏面酸化膜エッチングしたものを蒸着装置にセットした.
		\end{enumerate}

		\subsection{測定方法}
			\subsubsection{アルミゲート電極面積測定}
				\begin{enumerate}
					\item 測定前に標準器により,較正を行った.
					\item アルミ電極の大きさは2mm角や4mm角とされているが,今回は2mm角のアルミ電極に対して顕微鏡を用いて正確な電極面積を測定した.
					\item 実験時間の都合上,班員1名が測定したデータをそのまま測定値として結果に用いた.
				\end{enumerate}
			\subsubsection{容量測定}
				\begin{enumerate}
					\item グローブボックス内のプローブにウェーハをセットした.
					\item 定電圧電源により,酸化膜上のアルミ電極(ゲート)に外部電圧を加えた.\\印加電圧は±10Vの範囲で容量が大きく変化する間隔で測定を行った.
					\item 電圧印加時の静電容量をLCRメータを用いて測定した.
					\item 測定周波数1k, 10k,100k, 1Mの4種類を測定した.
				\end{enumerate}

	\section{実験結果}
		\subsection{アルミゲート電極面積測定}
			酸化時間ごとのウェーハの電極面積を表\refeq{tab:sankadenkyoku}に示す.
			\begin{table}[H]
			\begin{center}
			\caption{酸化時間ごとの電極面積}
			\label{tab:sankadenkyoku}
			\begin{tabular}{c|SSS} \toprule
				&\multicolumn{3}{c}{酸化時間\,[分]}\\
				&50&65&80\\ \hline
				横\,[$\mathrm{\mu m}$]&1963.58&1967.32&1966.92\\
				縦\,[$\mathrm{\mu m}$]&1971.53&1967.32&1969.34\\ \hline
				面積\,[n$\mathrm{m^{2}}$]&3871.26&3870.35&3873.53\\ \bottomrule
			\end{tabular}
			\end{center}
			\end{table}
			表\refeq{tab:sankadenkyoku}より,実際には1辺が2mm未満しかないことがわかる.
			加えて,容量測定などの操作によってアルミ電極の一部がはがれていることが確認できた.

		\subsection{容量測定}
			試料の酸化時間ごとの印加電圧による静電容量特性を表\refeq{tab:wehacap50},\refeq{tab:wehacap65},\refeq{tab:wehacap80}に示す.
			それぞれの電圧‐静電容量のグラフを図\refeq{fig:wehacap50},\refeq{fig:wehacap65},\refeq{fig:wehacap80}に示す.
\clearpage
			酸化時間が50分の試料の測定結果を表\refeq{tab:wehacap50}に示す.
			\begin{table}[H]
			\begin{center}
			\caption{酸化時間50分のウェーハの電圧‐容量特性}
			\label{tab:wehacap50}
			\begin{tabular}{S|SSSS} \toprule
				&\multicolumn{4}{c}{周波数\,[kHz]ごとの静電容量\,[nF]}\\ \hline
				印加電圧\,[V]&\multicolumn{1}{c}{1000}&\multicolumn{1}{c}{100}&\multicolumn{1}{c}{10}&\multicolumn{1}{c}{1}\\ \hline
				-10 & 1.51 & 1.45 & 1.44 & 1.44 \\
				-8 & 1.51 & 1.44 & 1.44 & 1.44 \\
				-6 & 1.50 & 1.43 & 1.43 & 1.43 \\
				-4 & 1.48 & 1.41 & 1.41 & 1.41 \\
				-3 & 1.44 & 1.39 & 1.38 & 1.38 \\
				-2.8 & 1.43 & 1.37 & 1.37 & 1.37 \\
				-2.6 & 1.41 & 1.35 & 1.35 & 1.35 \\
				-2.4 & 1.38 & 1.32 & 1.32 & 1.32 \\
				-2.2 & 1.33 & 1.28 & 1.28 & 1.28 \\
				-2 & 1.25 & 1.21 & 1.21 & 1.21 \\
				-1.8 & 1.11 & 1.09 & 1.10 & 1.10 \\
				-1.6 & 0.862 & 0.911 & 0.970 & 0.946 \\
				-1.4 & 0.639 & 0.66 & 0.754 & 0.718 \\
				-1.2 & 0.524 & 0.511 & 0.535 & 0.518 \\
				-1 & 0.457 & 0.444 & 0.499 & 0.453 \\
				0 & 0.350 & 0.344 & 0.345 & 0.680 \\
				1 & 0.350 & 0.343 & 0.344 & 0.710\\
				2 & 0.350 & 0.343 & 0.344 & 0.694 \\
				4 & 0.350 & 0.343 & 0.344 & 0.737 \\
				6 & 0.351 & 0.343 & 0.344 & 0.750 \\
				8 & 0.351 & 0.344 & 0.345 & 0.762 \\
				10 & 0.351 & 0.344 & 0.345 & 0.774 \\ 
				\bottomrule
			\end{tabular}
			\end{center}
			\end{table}

			表\refeq{tab:wehacap50}より,酸化時間50分の電圧‐静電容量特性のグラフを図\refeq{fig:wehacap50}に示す.

			\begin{figure}[H]
			\centering
			\includegraphics[width = 12cm]{figs/wehacap50.png}
			\caption{酸化時間50分のウェーハの電圧‐静電容量特性のグラフ}
			\label{fig:wehacap50}
			\end{figure}

			図\refeq{fig:wehacap50}より,印加電圧が-1V以上では,1kHzの時のみ静電容量が‐10Vのときの静電容量の2/5倍程度になっていることがわかる.
			それ以外の周波数では静電容量が大幅に減少したままになっている.
\clearpage
			酸化時間が65分の試料の測定結果を表\refeq{tab:wehacap65}に示す.
			\begin{table}[H]
			\begin{center}
			\caption{酸化時間65分のウェーハの電圧‐容量特性}
			\label{tab:wehacap65}
			\begin{tabular}{S|SSSS} 
				\toprule
				&\multicolumn{4}{c}{周波数\,[kHz]ごとの静電容量\,[nF]}\\ \hline
				印加電圧\,[V]&\multicolumn{1}{c}{1000}&\multicolumn{1}{c}{100}&\multicolumn{1}{c}{10}&\multicolumn{1}{c}{1}\\ \hline
				-10 & 1.26 & 1.21 & 1.20 & 1.20\\
				-8 & 1.26 & 1.20 & 1.20 & 1.20 \\
				-6 & 1.25 & 1.19 & 1.19 & 1.19 \\
				-4 & 1.23 & 1.17 & 1.18 & 1.18 \\
				-3 & 1.20 & 1.14 & 1.14 & 1.14 \\
				-2.8 & 1.18 & 1.13 & 1.13 & 1.13 \\
				-2.6 & 1.16 & 1.11 & 1.11 & 1.11 \\
				-2.4 & 1.12 & 1.08 & 1.08 & 1.08 \\
				-2.2 & 1.06 & 1.03 & 1.03 & 1.03 \\
				-2 & 0.967 & 0.962 & 0.969 & 0.970 \\
				-1.8 & 0.815 & 0.856 & 0.890 & 0.898 \\
				-1.6 & 0.634 & 0.68 & 0.767 & 0.817 \\
				-1.4 & 0.522 & 0.523 & 0.577 & 0.67 \\
				-1.2 & 0.458 & 0.448 & 0.459 & 0.537 \\
				-1 & 0.415 & 0.405 & 0.409 & 0.625 \\
				0 & 0.378 & 0.371 & 0.374 & 0.97 \\
				1 & 0.379 & 0.375 & 0.375 & 1.00 \\
				2 & 0.381 & 0.382 & 0.38 & 1.02 \\
				4 & 0.383 & 0.383 & 0.388 & 1.03 \\
				6 & 0.384 & 0.384 & 0.394 & 1.04 \\
				8 & 0.388 & 0.386 & 0.396 & 1.05 \\
				10 & 0.389 & 0.387 & 0.399 & 1.05 \\ 
				\bottomrule
			\end{tabular}
			\end{center}
			\end{table}

			表\refeq{tab:wehacap65}より,酸化時間65分の電圧‐静電容量特性のグラフを図\refeq{fig:wehacap65}に示す.
			\begin{figure}[H]
			\centering
			\includegraphics[width = 12cm]{figs/wehacap65.png}
			\caption{酸化時間65分のウェーハの電圧‐静電容量特性のグラフ}
			\label{fig:wehacap65}
			\end{figure}

			図\refeq{fig:wehacap65}より,印加電圧が-1V以上では,1kHzのときのみ静電容量が‐10Vのときの静電容量の3/4倍程度になっていることがわかる.
			それ以外の周波数では静電容量が大幅に減少したままになっている.
\clearpage
			酸化時間が80分の試料の測定結果を表\refeq{tab:wehacap80}に示す.
			\begin{table}[H]
			\begin{center}
			\caption{酸化時間80分のウェーハの電圧‐容量特性}
			\label{tab:wehacap80}
			\begin{tabular}{S|SSSS} \toprule
				&\multicolumn{4}{c}{周波数\,[kHz]ごとの静電容量\,[nF]}\\ \hline
				印加電圧\,[V]&\multicolumn{1}{c}{1000}&\multicolumn{1}{c}{100}&\multicolumn{1}{c}{10}&\multicolumn{1}{c}{1}\\ \hline
				-10 & 1.04 & 0.989 & 0.99 & 0.991 \\
				-8 & 1.04 & 0.985 & 0.988 & 0.989 \\
				-6 & 1.04 & 0.981 & 0.983 & 0.985 \\
				-4 & 1.02 & 0.97 & 0.972 & 0.974 \\
				-3 & 1.01 & 0.952 & 0.955 & 0.956 \\
				-2.8 & 0.999 & 0.947 & 0.948 & 0.949 \\
				-2.6 & 0.989 & 0.938 & 0.939 & 0.94 \\
				-2.4 & 0.975 & 0.924 & 0.925 & 0.926 \\
				-2.2 & 0.953 & 0.904 & 0.905 & 0.906 \\
				-2 & 0.919 & 0.872 & 0.872 & 0.873 \\
				-1.8 & 0.858 & 0.818 & 0.819 & 0.819 \\
				-1.6 & 0.751 & 0.733 & 0.739 & 0.741 \\
				-1.4 & 0.606 & 0.604 & 0.635 & 0.652 \\
				-1.2 & 0.495 & 0.481 & 0.503 & 0.545 \\
				-1 & 0.430 & 0.415 & 0.424 & 0.453 \\
				0 & 0.356 & 0.362 & 0.524 & 0.917 \\
				1 & 0.359 & 0.375 & 0.592 & 0.951 \\
				2 & 0.392 & 0.38 & 0.606 & 0.962 \\
				4 & 0.395 & 0.387 & 0.614 & 0.970 \\
				6 & 0.397 & 0.398 & 0.618 & 0.974 \\
				8 & 0.398 & 0.419 & 0.621 & 0.976 \\
				10 & 0.400 & 0.426 & 0.623 & 0.977 \\ \bottomrule
			\end{tabular}
			\end{center}
			\end{table}

			表\refeq{tab:wehacap80}より,酸化時間80分の電圧‐静電容量特性のグラフを図\refeq{fig:wehacap80}に示す.

			\begin{figure}[H]
			\centering
			\includegraphics[width = 12cm]{figs/wehacap80.png}
			\caption{酸化時間80分のウェーハの電圧‐静電容量特性のグラフ}
			\label{fig:wehacap80}
			\end{figure}

			図\refeq{fig:wehacap80}より,印加電圧が-1V以上では,1kHzのときにほぼ‐10Vのときの静電容量まで戻っていることがわかる.
			加えて,10kHzのときには‐10Vのときの静電容量の1/3倍程度まで戻っていることがわかる.

		\subsection{酸化膜厚の算出}
			ゲート電圧が‐10V(蓄積状態)のときのCより以下の式から酸化膜厚Tを求めた.
			測定した4種類の周波数(1MHz, 100kHz, 10kHz, 1kHz),3種類の酸化時間(50分, 65分, 80分)において導出した.
			\begin{equation}
				C = \epsilon_{OX}\,S/T = 3.9\,\epsilon_{0}\,S/T
			\end{equation}
			ここで,Sは電極面積,$\epsilon_{0} = 8.85\,\times\,10^{-12} [\mathrm{F/m}]$,$\mathrm{SiO_{2}}$の比誘電率は3.9とした.

			式を変形して,
			\begin{equation}
				T = 3.9\,\epsilon_{0}\,S/C
				\label{func:sankamaku}
			\end{equation}

			式\refeq{func:sankamaku}に代入して計算した結果を表\refeq{tab:sankamakukou}に示す.
			\begin{table}[H]
			\begin{center}
			\caption{酸化時間における各周波数での酸化膜厚}
			\label{tab:sankamakukou}
			\begin{tabular}{cc|cccc} \toprule
				\multicolumn{2}{c|}{酸化時間\,[分]}&50&65&80\\
				\multicolumn{2}{c|}{電極面積\,[$\mathrm{nm^{2}}$]}&3871.26&3870.35&3873.53\\ \hline
				&1000kHz&88.49&106.02&128.55\\
				酸化膜厚&100kHz&92.15&110.40&135.18\\
				T[nm]&10kHz&92.79&111.32&135.05\\
				&1kHz&92.79&111.32&134.91\\ \bottomrule
			\end{tabular}
			\end{center}
			\end{table}

			表\refeq{tab:sankamakukou}から酸化時間と酸化膜厚のグラフを図\refeq{fig:sankajikan-sankamakuatu}に示す.

			\begin{figure}[H]
			\centering
			\includegraphics[width = 10cm]{figs/sankamakuatu.png}
			\caption{酸化時間による酸化膜厚の変化}
			\label{fig:sankajikan-sankamakuatu}
			\end{figure}

			図\refeq{fig:sankajikan-sankamakuatu}より,酸化時間と酸化膜厚は比例関係にあり,酸化時間が増加するほど酸化膜厚は厚くなることがわかる.

\section{考察}
	\subsection{シリコンウェーハ洗浄方式}
		本実験でシリコンウェーハの初期洗浄に用いた洗浄方式はセミコクリーンによる超音波洗浄(ウェット洗浄)を行った.
		本節では,それ以外の洗浄方式について調査した.
		\cite{ref:洗浄方式}より,ウェーハ洗浄の種類は
		\begin{itemize}
			\item 化学的洗浄
			\item 物理的洗浄
		\end{itemize}
		の2種類に大別され,その中でも化学的洗浄では
		\begin{itemize}
			\item ウェット洗浄
			\item ドライ洗浄
		\end{itemize}
		の2種類に分けられる。ここで,ドライ洗浄について詳しく調べることにした.
		以下にそれぞれの洗浄方式の特徴を示す.
		\begin{enumerate}
			\item ドライ洗浄\\
				ウェット洗浄では,大量の薬液を使用・廃棄する必要があるなどの問題がある.
				それを解決するために考案された方式である.ドライ洗浄の例を以下に示す.
				\begin{itemize}
					\item 気相洗浄法\\
						気相洗浄法とは,化学薬品の気体や蒸気を用いてウェーハ表面を洗浄する方式のことである.
						基本的に,ウェーハ表面の不純物と気相状態の薬液との反応によって不純物が溶解・除去される.
						しかし,気体の密度が非常に低いため,反応を促進させるため,エネルギー源として熱,光,プラズマ,粒子線などを用いる.
					\item プラズマ洗浄\\
						プラズマ洗浄とは,ウェーハ表面に付着した不純物をプラズマで分解・気化することによって除去する方法である.
						酸素プラズマによって発生する酸素イオンなどは反応性が高く,ウェーハ上の不純物と反応して$\mathrm{CO_{2}}や\mathrm{H_{2}O}$を生成することで揮発除去される.
					\item 極低温エアロゾル洗浄\\
						極低温エアロゾル洗浄とは,極低温ガス中で凝固した固体粒子と不純物の衝突によってウェーハ上の不純物を物理的に除去する方法である.
						極低温(‐160~‐150度)付近で一定の圧力をかけた際にガスと液体の混合相を形成する気体を用いている.
					\item 超臨界流体洗浄\\
						回路パターンが微細な半導体では洗浄によるパターンの倒壊が発生する.
						これを解決するために,超臨界流体を用いて洗浄する方法である.
						\cite{ref:洗浄方式}によると超臨界状態において,流体はガスと同等の拡散性と液体同様の溶解力を有し,さらに表面張力がゼロになるという.
						これにより,複雑な回路パターンを持つ半導体でもパターン倒壊のリスクを減らすことができる.
				\end{itemize}
		\end{enumerate}
	\subsection{シリコンウェーハの色}
		酸化膜形成を行った際に,酸化時間によってシリコンウェーハの色が変色した.
		この現象について調査を行う.
		物理的に類似していると考えたのはCDの裏面やシャボン玉表面などで虹色の膜が見られる現象と同様のものだと考えた.
		それについて詳しく調べていく.

		類似していると仮定した現象は\cite{ref:薄膜干渉}によると,「薄膜干渉」と呼ばれ,薄膜表面で反射する光と薄膜で屈折し,その奥で反射した光が干渉して,青色などの様々な色を形成している.
		この薄膜干渉がシリコンウェーハとその表面に形成される酸化膜によって起きていると仮定する.
		表\refeq{tab:sankamakukou}からわかるように,酸化時間が変化すると,酸化膜厚が厚くなることから,酸化膜で屈折してシリコンウェーハで反射するまでの距離が長くなり,干渉する光の波長が短くなると考えられる.
		酸化時間が長いウェーハほど,すこし緑がかって見えたことから,今回の実験では仮定したものと同じ現象が起きていると予想する.
		したがって,薄膜干渉により,シリコンウェーハの色が変化したと考える.

	\subsection{ポジレジスト,ネガレジストの違い}
		ポジレジスト,ネガレジストの違いについて調査する.
		今回の実験で用いたのはポジ型レジストで,これは露光行程で光があたった箇所が水溶性になって溶解し,現像工程後になくなるレジストである.
		ネガ型レジストは逆に,露光行程で光があたった箇所が硬化し,水に不溶になり,現像工程後に残る.
		\cite{ref:ネガポジ}によると,昔はネガ型レジストの方が主流であったが,回路集積化によって微細なパターンの部分以外を固めるとなるとかなり高感度でなければいけないため,感度の点でポジ型に劣るということから現在ではポジ型レジストが主流と言われている.
	\subsection{C‐V特性の比較}
		理想C‐V特性である図\refeq{fig:MOSC-V}と実測値である図\refeq{fig:wehacap50},\refeq{fig:wehacap65},\refeq{fig:wehacap80}を比較してみる.
		理想特性では低周波の際に,減衰前の静電容量まで増えていることがわかる.そして,高周波の場合には全く増加しないことがわかる.
		図\refeq{fig:wehacap80}の1kHzのときのグラフではおおよそ理想通りに増加していることがわかる.
		しかし,図\refeq{fig:wehacap50},\refeq{fig:wehacap65}の1kHzのときのグラフでは理想よりも静電容量が小さくなっていることがわかる.
		この原因については,酸化時間によって酸化膜の厚さが変わることによって蓄えることのできる静電容量が異なると考えられる.
		酸化時間が長くなるほど形成される酸化膜は厚くなるので酸化膜とアルミ表面の距離が長くなることがわかる.
		これを静電容量の式$C=S/d$に対応させると,$d$が酸化膜とアルミ表面の距離と考えれば,酸化時間が長いほど静電容量は小さくなると予想できる.
		そのまま考えると,酸化時間が長いほど静電容量が小さくなっていくのだが,逆に,酸化時間が短すぎると,酸化膜の形成される面積が小さいのではないかと考察する.
		酸化膜が形成する面積と,酸化膜厚の増加の関係において,形成される面積の増加量が酸化膜厚の増加量よりも大きいうちは静電容量が増加する.
		反対に,形成される面積が酸化膜厚の増加量よりも小さくなれば,静電容量が小さくなっていくと考えられる.
		これにより,ある時間までは静電容量が時間とともに増加し,一定時間後に時間とともに静電容量が減少するような関係になっているのではないかと考察した.
	
	\subsection{酸化膜厚の比較}
		\cite{ref:指導書}の図1.6と本報告書の図\refeq{fig:sankajikan-sankamakuatu}を比較する.
		\cite{ref:指導書}の図1.6では両対数グラフが用いられており,図\refeq{fig:sankajikan-sankamakuatu}とは表示が異なるが,
		値としては酸化温度が1000度のグラフから,酸化時間が50分のとき,酸化膜厚は0.2$\mathrm{\mu m}$,酸化時間80分のとき,酸化膜厚は0.3$\mathrm{\mu m}$を示していた.
		実測値は表\refeq{tab:sankamakukou}から,酸化時間が50分で測定周波数が1kHzのとき,酸化膜厚は0.092$\mathrm{\mu m}$,酸化時間80分のとき,酸化膜厚は0.135$\mathrm{\mu m}$を示していた.
		比較すると,それぞれ約0.1$\mathrm{\mu m}$の差があることがわかる.
		これは恐らく,電極面積の差や測定周波数,酸化温度による誤差であると考える.
		電極面のアルミが一部削れていたり酸化温度が一定ではなかったことが静電容量や電極面積に影響を与えたのではないかと予想する.
		しかし,酸化時間によって酸化膜厚が増加するというグラフの傾向は同じであったため,今回の実験はおおむね理想特性通りの特性を得られることができたと思われる.

\section{所感}
	今回の実験で初めて神戸高専のクリーンルームに入ることができた.
	以前インターンで製薬会社のクリーンルームに入る機会があったが,それと比較するとかなり管理のクラスは低くなっていると実感した.
	クリーンルーム内の装置(特に電気炉)がそこそこ年季が入っているという印象を受けた.
	それ以外では,MOS構造の作成行程でシリコンウェーハの色が酸化時間によって変化することがとても興味深く,それぞれの手順の時間設定がどのような根拠に基づいて実験を行っているのかが個人的に気にかかる部分であった.
\begin{thebibliography}{99}
\bibitem{ref:指導書}
西 敬生「実験実習指導書」神戸高専電子工学科 pp.09 - 14

\bibitem{ref:洗浄方式}
Semi journal「Siウェハーの洗浄法」閲覧日 2023/06/25\\
https://semi\-journal.jp/basics/si\-chem/cleaning\-2.html

\bibitem{ref:薄膜干渉}
ITES「恋する半導体 『可視光線編』」閲覧日 2023/06/26\\
https://www.ites.co.jp/wafer/semikoi\_05.html

\bibitem{ref:ネガポジ}
永田 敬「フォトリソグラフィ」閲覧日 2023/06/26\\
http://www.osakac.ac.jp/labs/matsuura/japanese/lecture/semicondic/ha/ha021.pdf

\end{thebibliography}
\end{document}
