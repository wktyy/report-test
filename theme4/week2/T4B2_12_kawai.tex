\documentclass{ltjsarticle}
%\usepackage[dvipdfmx]{graphicx}
\usepackage{graphicx}
\usepackage{booktabs}
\usepackage{mathcomp}
\usepackage{array}
\usepackage{mathtools,amssymb}
\usepackage{siunitx}
\usepackage{multirow}
\usepackage{tabularx}
\usepackage{subcaption}
\usepackage{float}
\usepackage{setspace}

\usepackage{listings,jvlisting}
\lstset{
	basicstyle={\ttfamily},
	identifierstyle={\small},
	commentstyle={\smallitshape},
	keywordstyle={\small\bfseries},
	ndkeywordstyle={\small},
	stringstyle={\small\ttfamily},
	frame={tb},
	breaklines=true,
	columns=[l]{fullflexible},
	numbers=left,
	numberstyle={\scriptsize},
	stepnumber=1,
	lineskip=-0.5ex,
	tabsize=2
}
\renewcommand{\lstlistingname}{ソースコード}

\usepackage{kcctd-report}


\title{各種計算ハードウェアの活用\,VHDLによるディジタル回路の設計}
\adviser{木場 隼介 先生}

\sdate{令和5年10月12日}
\edate{令和5年10月19日}
\fdate{令和5年10月24日}
\rdate{}

\grade{5}
\anumber{12}
\gnumber{B}
\name{河合 将暉}
\jname{}
\comment{}
\begin{document}
\maketitle

\section{目的}
本実験では,業界標準のVHDLとAltera(Intel)社のQuartus Prime Lite
Editionを使用し,HDL,FPGAを用いたディジタル回路設計の基本的な
考え方と手法を習得することを目的とする。
\section{解説}
	\subsection{VHDL・FPGAとは}
	VHDLとはIEEEで標準化されたデジタル回路設計用のハードウェア記述言語(HDL:Hardware Description Language)
	である。従来の電子回路設計はプリント回路基板設計用のCADなどを用いて多数の
	電子部品を回路図記号で表記することが一般的で製造後に回路構成を変更できなか
	ったが,現場で論理回路の構成をプログラムできる論理回路を集積したデバイス(FPGA:Field Programable Gate Array)
	が登場すると,HDLを用いてその論理ゲートをプログラミングのように記述することが
	可能になった.
	HDLにはVHDL,VerilogHDLの2種類が存在し,VHDLはFPGAが登場した初期から存在し,
	Ada言語やPascal言語を参考に記法が作られている。VerilogHDLは比較的新しい言語で
	C言語ベースの記法で作られている。明確な違いの例を挙げると,論理演算子が
	VHDLではand or not であるが VerilogHDLでは \& \textbar \, \textasciitilde などの
	記号で表されている。
	
	\subsection{Altera社について}
	Altera社は1983年に設立されたPLD(Programable Logic Device)の代表的企業で,
	システムオンプログラマブルチップを可能とするべく,様々な技術を開発し,その中
	ではチップ内にメモリやマイクロプロセッサ,トランシーバを埋め込んだものも存在
	するという。現在ではIntel社に買収され,FPGA部門として活動している。
	\subsection{Quartus Primeとは}
	

\section{実験内容}
	\subsection{使用器具}
		\begin{table}[H]
		\centering
		\caption{test}
		\label{tab:used}
		\begin{tabular}{clllll} \toprule
		No&\multicolumn{1}{l}{機器名}&\multicolumn{1}{l}{型番}&\multicolumn{1}{l}{シリアルNo}&\multicolumn{1}{l}{備考}\\ \hline
		1&FPGAボード&Cyclone V E FPGA&2&シリアルNoは\\
		&&Development Kit&&外箱の番号を記載\\
		2&PC&ASUS &&\\
		\bottomrule
		\end{tabular}
		\end{table}
	\subsection{実験準備}
		Altera(Intel)社から発売されている評価ボードCyclone V E FPGA Development Kit を使用した。
		このボードに搭載されているFPGAのCyclone V E FPGA (5CEFA7F31I7N)と周辺機器のLED4個,
		押しボタンスイッチ4個,クロック発振器(50MHz),キャラクタ液晶などが搭載されている。
		準備として,Quartus Primeの操作手順について以下に示す。
		\begin{enumerate}
			\item New Project Wizardの起動
			\item プロジェクト名の指定
			\item テンプレートの設定
			\item 使用する設計ファイルの設定
			\item ターゲットデバイスファミリの指定
			\item EDA Toolの指定
			\item 設定の確認
		\end{enumerate}
		これらの設定を行ったあと,カウンタの数字によって点灯するLEDを変えるプログラムを保存した。
		そして,HDLで記述された回路は論理合成と配置配線を行うことでLSIに実装できるようになる.
		FPGA設計用のツールでは,これらの処理はひとまとめにされ,コンパイルと呼ばれている.
		以下にコンパイルの処理手順について示す.
		\begin{enumerate}
			\item プリフィット
			\item ピン配置
			\item ポストフィット
		\end{enumerate}
		上記によりコンパイルが行われる.
		次に,コンパイル後に出力されたコンフィグレーションデータをFPGAボードに書き込む.
		この際に,Windowのツールバー上「Tools」→「Programmer」を選択し,Programmerを開く.
		ここで書き込むデバイスの指定を行い,書き込みを行った.

		以上の準備を以降の実験でも同様に行い,基本的な論理回路の動作確認をした.

	\subsection{LEDの点灯と消灯}
		FPGAに搭載されているLEDはFPGAのI/Oピンが``L''のときにLEDが点灯し,``H''のときに
		消灯する.
		以下にSample1のソースコードをソースコード\refeq{code:sample1}に示す.

		\begin{lstlisting}[caption=sample1, label=code:sample1]
			testtest
			titless
		\end{lstlisting}

		FPGAのピン配置を以下の表\refeq{tab:LED1}に示す.

		\begin{table}[H]
		\begin{center}
		\caption{LED点灯回路のピン配置}
		\label{tab:LED1}
		\begin{tabular}{cc} \toprule
			portの名前 & ピン名称 \\ \hline
			led\_out & PIN\_AK3\\
		\bottomrule
		\end{tabular}
		\end{center}
		\end{table}

		ここで,VHDL記述の基本事項として,ライブラリ宣言部とパッケージ宣言部で始まり,
		エンティティ宣言で入出力ポートを定義し,アーキテクチャ宣言に動作を記述していく.
		コメントアウトは``--''で記述できる.
		以下に各事項についての詳細を示す.
	
		\begin{enumerate}
			\item ライブラリ宣言
			\item パッケージ呼び出し
			\item エンティティ宣言
			\item アーキテクチャ宣言
		\end{enumerate}

	\subsection{LEDでのバイナリ表示}
		3.3項で行った一つのLED表示を応用して,複数のLEDを制御可能なプログラムを作成した.
		そして,4つのLEDを用いてバイナリ表示を表現した.
		以下にSample2のソースコードをソースコード\refeq{code:sample2}に示す.

		\begin{lstlisting}[caption=sample2, label=code:sample2]
			aaa
		\end{lstlisting}

		FPGAのピン配置を以下の表\refeq{tab:LED4}に示す.

		\begin{table}[H]
		\begin{center}
		\caption{4つのLEDで2進数表現する回路のピン配置}
		\label{tab:LED4}
		\begin{tabular}{cc} \toprule
			portの名前 & ピン名称 \\ \hline
			led\_out[0] & PIN\_AK3 \\
			led\_out[1] & PIN\_AJ4 \\
			led\_out[2] & PIN\_AJ5 \\
			led\_out[3] & PIN\_AK6 \\
			\bottomrule
		\end{tabular}
		\end{center}
		\end{table}

	\subsection{LEDでのグレイ符号表示}
	\subsection{LEDのクロック同期動作}
	\subsection{FPGAの論理演算}
	\subsection{I/O機器によるLED点消灯}
	\subsection{2進数ルーレット}
	\subsection{4色スロットマシン}
\section{実験結果}
\section{考察}
	\subsection{分周回路}
\begin{thebibliography}{99}
\bibitem{ref:指導書}
「各種計算ハードウェアの活用~VHDLによるディジタル回路の設計~」
神戸高専電子工学科 pp.01-33

\bibitem{ref:VerilogVHDL}
トーマスイッチ「VerilogとVHDLの違いとはわかりやすく解説」
https://toumaswitch.com/5q0shy1sod/
\end{thebibliography}
\end{document}