\documentclass{ltjsarticle}
%\usepackage[dvipdfmx]{graphicx}
\usepackage{graphicx}
\usepackage{booktabs}
\usepackage{mathcomp}
\usepackage{array}
\usepackage{mathtools,amssymb}
\usepackage{siunitx}
\usepackage{multirow}
\usepackage{tabularx}
\usepackage{subcaption}
\usepackage{float}
\usepackage{setspace}

\usepackage{listings,jvlisting}
\lstset{
	basicstyle={\ttfamily},
	identifierstyle={\small},
	commentstyle={\smallitshape},
	keywordstyle={\small\bfseries},
	ndkeywordstyle={\small},
	stringstyle={\small\ttfamily},
	frame={tb},
	breaklines=true,
	columns=[l]{fullflexible},
	numbers=left,
	numberstyle={\scriptsize},
	stepnumber=1,
	lineskip=-0.5ex,
	tabsize=2
}
\renewcommand{\lstlistingname}{ソースコード}

\usepackage{kcctd-report}


\title{各種計算ハードウェアの活用\,VHDLによるディジタル回路の設計}
\adviser{木場 隼介 先生}

\sdate{令和5年10月12日}
\edate{令和5年10月19日}
\fdate{令和5年10月24日}
\rdate{}

\grade{5}
\anumber{12}
\gnumber{B}
\name{河合 将暉}
\jname{}
\comment{}
\begin{document}
\maketitle

\section{目的}
本実験では,業界標準のVHDLとAltera(Intel)社のQuartus Prime Lite
Editionを使用し,HDL,FPGAを用いたディジタル回路設計の基本的な
考え方と手法を習得することを目的とする。
\section{解説}
	\subsection{VHDL・FPGAとは}
	VHDLとはIEEEで標準化されたデジタル回路設計用のハードウェア記述言語(HDL:Hardware Description Language)
	である。従来の電子回路設計はプリント回路基板設計用のCADなどを用いて多数の
	電子部品を回路図記号で表記することが一般的で製造後に回路構成を変更できなか
	ったが,現場で論理回路の構成をプログラムできる論理回路を集積したデバイス(FPGA:Field Programable Gate Array)
	が登場すると,HDLを用いてその論理ゲートをプログラミングのように記述することが
	可能になった.
	HDLにはVHDL,VerilogHDLの2種類が存在し,VHDLはFPGAが登場した初期から存在し,
	Ada言語やPascal言語を参考に記法が作られている\cite{ref:VerilogVHDL}。VerilogHDLは比較的新しい言語で
	C言語ベースの記法で作られている\cite{ref:VerilogVHDL}。明確な違いの例を挙げると,論理演算子が
	VHDLではand or not であるが VerilogHDLでは \& \textbar \, \textasciitilde などの
	記号で表されている。
	
	\subsection{Altera社について}
	Altera社は1983年に設立されたPLD(Programable Logic Device)の代表的企業で,
	システムオンプログラマブルチップを可能とするべく,様々な技術を開発し,その中
	ではチップ内にメモリやマイクロプロセッサ,トランシーバを埋め込んだものも存在
	するという。現在ではIntel社に買収され,FPGA部門として活動している。
	\subsection{Quartus Primeとは}
	

\section{実験内容}
	\subsection{使用器具}
		\begin{table}[H]
		\centering
		\caption{test}
		\label{tab:used}
		\begin{tabular}{clllll} \toprule
		No&\multicolumn{1}{l}{機器名}&\multicolumn{1}{l}{型番}&\multicolumn{1}{l}{シリアルNo}&\multicolumn{1}{l}{備考}\\ \hline
		1&FPGAボード&Cyclone V E FPGA&2&シリアルNoは\\
		&&Development Kit&&外箱の番号を記載\\
		2&PC&ASUS &&\\
		\bottomrule
		\end{tabular}
		\end{table}
	\subsection{実験準備}
		Altera(Intel)社から発売されている評価ボードCyclone V E FPGA Development Kit を使用した。
		このボードに搭載されているFPGAのCyclone V E FPGA (5CEFA7F31I7N)と周辺機器のLED4個,
		押しボタンスイッチ4個,クロック発振器(50MHz),キャラクタ液晶などが搭載されている。
		準備として,Quartus Primeの操作手順について以下に示す。
		\begin{enumerate}
			\item New Project Wizardの起動
			\item プロジェクト名の指定
			\item テンプレートの設定
			\item 使用する設計ファイルの設定
			\item ターゲットデバイスファミリの指定
			\item EDA Toolの指定
			\item 設定の確認
		\end{enumerate}
		これらの設定を行ったあと,カウンタの数字によって点灯するLEDを変えるプログラムを保存した。
		そして,HDLで記述された回路は論理合成と配置配線を行うことでLSIに実装できるようになる.
		FPGA設計用のツールでは,これらの処理はひとまとめにされ,コンパイルと呼ばれている.
		以下にコンパイルの処理手順について示す.
		\begin{enumerate}
			\item プリフィット
			\item ピン配置
			\item ポストフィット
		\end{enumerate}
		上記によりコンパイルが行われる.
		次に,コンパイル後に出力されたコンフィグレーションデータをFPGAボードに書き込む.
		この際に,Windowのツールバー上「Tools」→「Programmer」を選択し,Programmerを開く.
		ここで書き込むデバイスの指定を行い,書き込みを行った.

		以上の準備を以降の実験でも同様に行い,基本的な論理回路の動作確認をした.

	\subsection{LEDの点灯と消灯}
		FPGAに搭載されているLEDはFPGAのI/Oピンが``L''のときにLEDが点灯し,``H''のときに
		消灯する.LEDを点灯・消灯させるプログラムをソースコード\refeq{code:sample1}に示す.

		\begin{lstlisting}[caption=sample1, label=code:sample1]
			library ieee;
			use ieee.std_logic_1164.all;

			entity sample01 is

			port (
			led_out : out std_logic);

			end sample01;

			architecture rtl of sample01 is

			begin

			-- LEDはLOWで点灯するように回路が構成されているため
			-- std_logic型の0を出力する。これを1にすると消灯する。
			led_out <= '0';

			end rtl;
		\end{lstlisting}

		FPGAのピン配置を表\refeq{tab:LED1}に示す.

		\begin{table}[H]
		\begin{center}
		\caption{LED点灯回路のピン配置}
		\label{tab:LED1}
		\begin{tabular}{cc} \toprule
			portの名前 & ピン名称 \\ \hline
			led\_out & PIN\_AK3\\
		\bottomrule
		\end{tabular}
		\end{center}
		\end{table}

		ここで,VHDL記述の基本事項として,ライブラリ宣言部とパッケージ宣言部で始まり,
		エンティティ宣言で入出力ポートを定義し,アーキテクチャ宣言に動作を記述していく.
		コメントアウトは``--''で記述できる.
		各事項についての詳細を以下に示す.
	
		\begin{enumerate}
			\item ライブラリ宣言
			\item パッケージ呼び出し
			\item エンティティ宣言
			\item アーキテクチャ宣言
		\end{enumerate}

	\subsection{LEDでのバイナリ表示}
		3.3項で行った一つのLED表示を応用して,複数のLEDを制御可能なプログラムを作成した.
		そして,4つのLEDを用いてバイナリ表示を表現した.
		ソースコード\refeq{code:sample2}を以下に示す.

		\begin{lstlisting}[caption=sample2, label=code:sample2]
			library ieee;
			use ieee.std_logic_1164.all;

			entity sample02 is

			port (
			led_out : out std_logic_vector(3 downto 0));

			end sample02;

			architecture rtl of sample02 is

			begin

			-- 10進数の5を表現する。LEDはLOWで点灯するため、
			-- ビット反転した値を出力する
			led_out <= "1010";

			end rtl;

		\end{lstlisting}

		FPGAのピン配置を表\refeq{tab:LED4}に示す.

		\begin{table}[H]
		\begin{center}
		\caption{4つのLEDで2進数表現する回路のピン配置}
		\label{tab:LED4}
		\begin{tabular}{cc} \toprule
			portの名前 & ピン名称 \\ \hline
			led\_out[0] & PIN\_AK3 \\
			led\_out[1] & PIN\_AJ4 \\
			led\_out[2] & PIN\_AJ5 \\
			led\_out[3] & PIN\_AK6 \\
			\bottomrule
		\end{tabular}
		\end{center}
		\end{table}

		上記のプログラムを実行した結果を確認し,点灯パターンが異なる場合についても動作確認を行った.
		この際,4ビットの信号はstd\_logic\_vectorによってバスとして定義している.
		そのため,led\_out <= "1010";のようにまとめて記述することが可能である.

	\subsection{LEDでのグレイ符号表示}
		本項では,2進数の進み方が1ビットずつしか変化しないグレイ符号を表示するプログラムを作成した.
		ソースコード\refeq{code:sample3}を以下に示す.
		\begin{lstlisting}[caption=sample3, label=code:sample3]
			library ieee;
			use ieee.std_logic_1164.all;

			entity sample03 is

			port (
			led_out : out std_logic_vector(3 downto 0));

			end sample03;

			architecture rtl of sample03 is

			signal counter : std_logic_vector(3 downto 0):= (others=>'0');

			begin

			counter <= "0010";                -- ①

			process (counter)                 -- ②
			begin
			case counter is
			when "0000" =>
			led_out <= "1111";          -- ③
			when "0001" =>
			led_out <= "1110";
			when "0010" =>
			led_out <= "1100";
			when "0011" =>
			led_out <= "1101";
			when "0100" =>
			led_out <= "1001";
			when "0101" =>
			led_out <= "1000";
			when "0110" =>
			led_out <= "1010";
			when "0111" =>
			led_out <= "1011";
			when "1000" =>
			led_out <= "0011";
			when "1001" =>
			led_out <= "0010";
			when "1010" =>
			led_out <= "0000";
			when "1011" =>
			led_out <= "0001";
			when "1100" =>
			led_out <= "0101";
			when "1101" =>
			led_out <= "0100";
			when "1110" =>
			led_out <= "0110";
			when "1111" =>
			led_out <= "0111";
			when others => null;
			end case;
			end process;
			end rtl;

		\end{lstlisting}

		上記のプログラムで2進数がグレイ符号に変換して出力されているか確認した.
		また,それぞれ異なる数字を変換した場合やグレイ符号を2進数に変換した場合についても
		プログラムを変更し,同様に確認した.

	\subsection{LEDのクロック同期動作}
		sample3で使用したグレイ符号へのエンコーダ/デコーダ回路を自動的に一定周期で変化させるプログラムを
		ソースコード\refeq{code:sample4}に示す.
		\begin{lstlisting}[caption=sample4, label=code:sample4]
			library ieee;
			use ieee.std_logic_1164.all;
			use ieee.std_logic_unsigned.all;

			entity sample04 is

			port (
			clk     : in  std_logic;
			led_out : out std_logic_vector(3 downto 0));

			end sample04;

			architecture rtl of sample04 is

			signal counter : std_logic_vector(3 downto 0);

			-- 以下は分周用のカウンタとクロック
			signal div_counter : std_logic_vector(25 downto 0);
			signal div_clk : std_logic;

			begin

			-- 以下は50MHzのクロックから1秒弱のクロックを作成する
			-- 分周回路です。
			process (clk)
			begin  -- process
			if clk'event and clk = '1' then
			div_counter <= div_counter + 1;
			end if;
			end process;

			-- div_clkが1Hzのクロックで、デコーダへのクロック入力
			div_clk <= div_counter(19);
			-- ここまでが分周回路

			process (div_clk)
			begin
			if div_clk'event and div_clk = '1' then
			counter <= counter + 1;
			end if;
			end process;

			process (counter)
			begin
			case counter is
			when "0000" =>
			led_out <= "1111";
			when "0001" =>
			led_out <= "1110";
			when "0010" =>
			led_out <= "1100";
			when "0011" =>
			led_out <= "1101";
			when "0100" =>
			led_out <= "1001";
			when "0101" =>
			led_out <= "1000";
			when "0110" =>
			led_out <= "1010";
			when "0111" =>
			led_out <= "1011";
			when "1000" =>
			led_out <= "0011";
			when "1001" =>
			led_out <= "0010";
			when "1010" =>
			led_out <= "0000";
			when "1011" =>
			led_out <= "0001";
			when "1100" =>
			led_out <= "0101";
			when "1101" =>
			led_out <= "0100";
			when "1110" =>
			led_out <= "0110";
			when "1111" =>
			led_out <= "0111";
			when others => null;
			end case;
			end process;
			end rtl;
		\end{lstlisting}

		上記プログラムを実行し、一定周期で1ビットずつLEDが動作することを確認した。
		また、分周回路のクロックを変化させることでLEDが点滅する速度が変わることを確認した。

	\subsection{FPGAの論理演算}
		HDLによって記述された論理回路をプッシュボタンとLEDで確認するプログラムをソースコード\refeq{code:sample5}に示す。
		\begin{lstlisting}[caption=sample5, label=code:sample5]
			library ieee;
			use ieee.std_logic_1164.all;

			entity sample05 is

			port (
			led_out : out std_logic;
			pb0: in std_logic;
			pb1: in std_logic);

			end sample05;

			architecture rtl of sample05 is

			signal operation_result : std_logic;
			signal pb0_positive : std_logic;
			signal pb1_positive : std_logic;

			begin

			pb0_positive <= not pb0;  -- スイッチ入力は負論理なので not で正論理に変換
			pb1_positive <= not pb1;  -- スイッチ入力は負論理なので not で正論理に変換

			operation_result <= (pb0_positive or pb1_positive);

			led_out <= not operation_result;   -- LEDは負論理動作なので not で正論理から変換

			end rtl;
		\end{lstlisting}

		\begin{table}[H]
		\begin{center}
		\caption{sample5のピン配置}
		\label{tab:sample5}
		\begin{tabular}{cc} \toprule
			port名&ピン名称\\
			led\_out&PIN\_AK3\\
			pb0&PIN\_AB12\\
			pb1&PIN\_AB13\\
			\bottomrule
		\end{tabular}
		\end{center}
		\end{table}

	\subsection{I/O機器によるLED点消灯}

		\begin{lstlisting}[caption=sample6, label=sample6]
			library ieee;
			use ieee.std_logic_1164.all;
			use ieee.std_logic_unsigned.all;

			entity sample06 is

			-- div_bitsは分周回路に用いられているカウンタのビット数
			generic (
			div_bits : integer := 16);

			port (
			clk     : in  std_logic;
			sw_in   : in  std_logic;
			led_out : out std_logic);

			end sample06;

			architecture rtl of sample06 is

			signal led_node : std_logic := '0';
			signal div_counter : std_logic_vector(div_bits-1 downto 0) := (others => '0');
			signal sw_in_node : std_logic;
			signal sw_latch_on : std_logic := '0';
			begin

			-- 入力クロックを分周し、チャタリングを除去する回路
			process (clk)
			begin
			if clk'event and clk = '1' then  -- rising clock edge
			div_counter <= div_counter + 1;
			end if;
			end process;

			-- 分周したクロックでスイッチからの入力をラッチする回路
			-- 分周用カウンタの最上位ビットをクロックとして用いる
			process (div_counter(div_bits-1))
			begin  -- process
			if div_counter(div_bits-1)'event and div_counter(div_bits-1) = '1' then
			sw_in_node <= sw_in;
			end if;
			end process;

			-- 基本クロックに同期してDFF(led_node)をトグルする。
			-- チャタリングが発生すると、この回路が複数回アクティブになり、
			-- パルスが何度も発生し、LEDがついたまま、あるいは、
			-- 消えたままの場合がある。このような時はdiv_counterのビット数を増やす。
			process (clk)
			begin  -- process
			if clk'event and clk = '1' then  -- rising clock edge
			if sw_in_node = '0' and sw_latch_on = '0' then
			led_node <= not led_node;
			sw_latch_on <= '1';
			elsif sw_in_node = '1' and sw_latch_on = '1' then
			sw_latch_on <= '0';
			elsif sw_in_node = '0' and sw_latch_on = '1' then
			sw_latch_on <= '1';
			end if;
			end if;
			end process;

			led_out <= led_node;

			end rtl;
		\end{lstlisting}

	\subsection{2進数ルーレット}

		\begin{lstlisting}[caption=sample7, label=sample7]
			library ieee;
			use ieee.std_logic_1164.all;
			use ieee.std_logic_unsigned.all;

			entity sample07 is
			generic (
			div_bits : integer := 15);

			port (
			clk     : in  std_logic;
			sw_in   : in  std_logic;
			led_out : out std_logic_vector(3 downto 0));

			end sample07;

			architecture rtl of sample07 is

			-- ステートマシンのためのノード宣言
			signal counter_curr, counter_next : std_logic_vector(3 downto 0):= (others=>'0');

			-- チャタリング除去のための分周クロックとスイッチ入力のためのラッチ
			signal clk_div_sw, sw_node, sw_dff : std_logic := '0';
			-- LED出力用の分周カウンタ。このカウンタのビット数を増やすと
			-- ルーレットの変化がゆっくりになります。
			signal clk_div_count : std_logic_vector(22 downto 0) := (others => '0');
			signal div_counter : std_logic_vector(div_bits-1 downto 0) := (others => '0');
			signal sw_latch_on : std_logic := '0';
			begin

			-- チャタリング除去のためのクロック分周
			process (clk)
			begin
			if clk'event and clk = '1' then
			div_counter <= div_counter + 1;
			end if;
			end process;
			-- 最上位をチャタリング除去クロックとして用いる
			clk_div_sw <=  div_counter(div_bits-1);

			-- チャタリング除去してスイッチ入力値を採る
			process (clk_div_sw)
			begin
			if clk_div_sw'event and clk_div_sw = '1' then
			sw_node <= sw_in;
			end if;
			end process;

			-- ルーレットの回る速さをゆっくりにするための分周回路
			-- clk_div_countの最上位ビットをステートマシンのクロックとして用います。
			process (clk)
			begin
			if clk'event and clk = '1' then
			clk_div_count <= clk_div_count + 1;
			end if;
			end process;

			-- ルーレットをストップ・スタートさせるためのラッチ
			-- スイッチ入力によってトグルされます。
			process (clk)
			begin  -- process
			if clk'event and clk = '1' then  -- rising clock edge
			if sw_node = '0' and sw_latch_on = '0' then
			sw_dff <= not sw_dff;
			sw_latch_on <= '1';
			elsif sw_node = '1' and sw_latch_on = '1' then
			sw_latch_on <= '0';
			elsif sw_node = '0' and sw_latch_on = '1' then
			sw_latch_on <= '1';
			end if;
			end if;
			end process;

			-- LED出力を制御するステートマシン
			process (clk)
			begin
			if clk_div_count(22)'event and clk_div_count(22) = '1' then  -- rising clock edge
			if sw_dff = '1' then
			counter_curr <= counter_next;
			end if;
			end if;
			end process;

			process (counter_curr)
			begin
			case counter_curr is
			when "1011" =>
			counter_next <= "0000";         -- (a)
			when "0000" =>
			counter_next <= "0001";
			when "0001" =>
			counter_next <= "0010";
			when "0010" =>
			counter_next <= "0011";
			when "0011" =>
			counter_next <= "0100";
			when "0100" =>
			counter_next <= "0101";
			when "0101" =>
			counter_next <= "0110";
			when "0110" =>
			counter_next <= "0111";
			when "0111" =>
			counter_next <= "1000";
			when "1000" =>
			counter_next <= "1001";
			when "1001" =>
			counter_next <= "1010";
			when "1010" =>
			counter_next <= "1011";
			when others => null;
			end case;
			end process;


			process (counter_curr)
			begin
			led_out <= not counter_curr;
			end process;
			end rtl;

		\end{lstlisting}

		\subsection{4色スロットマシン}

			\begin{lstlisting}[caption=sample8, label=sample8]
				library ieee;
				use ieee.std_logic_1164.all;
				use ieee.std_logic_unsigned.all;

				entity sample08 is
				generic (
				div_bits : integer := 15);

				port (
				clk     : in  std_logic;
				resetn   : in  std_logic;
				sw_in   : in  std_logic;
				led_out0, led_out1, led_out2 : out std_logic_vector(3 downto 0));

				end sample08;

				architecture rtl of sample08 is

				component roulette

				port (
				clk     : in  std_logic;
				resetn   : in  std_logic;
				sw_in   : in  std_logic;
				led_out : out std_logic_vector(3 downto 0));

				end component;

				signal clk_div, clk_roulette, sw_node : std_logic := '0';
				signal div_counter : std_logic_vector(22 downto 0) := (others => '0');
				signal sw0, sw1, sw2 : std_logic;
				signal sw_latch, sw_latch_on : std_logic := '0';
				begin

				roulette_unit0 : roulette port map (
				clk     => clk_roulette,
				resetn   => resetn,
				sw_in   => sw0,
				led_out => led_out0);

				roulette_unit1 : roulette port map (
				clk     => clk_roulette,
				resetn   => resetn,
				sw_in   => sw1,
				led_out => led_out1);

				roulette_unit2 : roulette port map (
				clk     => clk_roulette,
				resetn   => resetn,
				sw_in   => sw2,
				led_out => led_out2);

				process (clk)
				begin
				if clk'event and clk = '1' then  -- rising clock edge
				div_counter <= div_counter + 1;
				end if;
				end process;

				clk_div <=  div_counter(div_bits-1);
				clk_roulette <= div_counter(22);

				-- チャタリング除去
				process (clk_div)
				begin  -- process
				if clk_div'event and clk_div = '1' then  -- rising clock edge
				sw_node <= sw_in;
				end if;
				end process;

				-- スイッチ入力を1パルスにして出力する回路
				process (clk)
				begin  -- process
				if clk'event and clk = '1' then  -- rising clock edge
				if sw_node = '0' and sw_latch_on = '0' then
				sw_latch <= '1';
				sw_latch_on <= '1';
				elsif sw_node = '1' and sw_latch_on = '1' then
				sw_latch <= '0';
				sw_latch_on <= '0';
				elsif sw_node = '0' and sw_latch_on = '1' then
				sw_latch <= '0';
				sw_latch_on <= '1';
				end if;
				end if;
				end process;

				-- スイッチによるパルスを受け取ってルーレットを止めていく回路
				process (clk, resetn)
				begin  -- process
				if resetn = '0' then
				sw0 <= '1';
				sw1 <= '1';
				sw2 <= '1';
				elsif clk'event and clk = '1' then  -- rising clock edge
				if sw_latch = '1' then
				if (sw0 = '1' and sw1 = '1' and sw2 = '1') then
				sw0 <= '0';
				elsif (sw0 = '0' and sw1 = '1' and sw2 = '1') then
				sw1 <= '0';
				elsif (sw0 = '0' and sw1 = '0' and sw2 = '1') then
				sw2 <= '0';
				end if;
				end if;
				end if;
				end process;
				end rtl;

			\end{lstlisting}



\begin{thebibliography}{99}
\bibitem{ref:指導書}
「各種計算ハードウェアの活用~VHDLによるディジタル回路の設計~」
神戸高専電子工学科 pp.01-33

\bibitem{ref:VerilogVHDL}
トーマスイッチ「VerilogとVHDLの違いとはわかりやすく解説」
https://toumaswitch.com/5q0shy1sod/
\end{thebibliography}
\end{document}