\documentclass{ltjsarticle}
%\usepackage[dvipdfmx]{color}
\usepackage{booktabs}
\usepackage{mathcomp}
\usepackage{array}
\usepackage{mathtools,amssymb}
\usepackage{siunitx}
\usepackage{multirow}
\usepackage{tabularx}
\usepackage{subcaption}
\usepackage{float}
\usepackage{kcctd-report}
\usepackage{listings,jvlisting}
\lstset{
	basicstyle={\ttfamily},
	identifierstyle={\small},
	commentstyle={\smallitshape},
	keywordstyle={\small\bfseries},
	ndkeywordstyle={\small},
	stringstyle={\small\ttfamily},
	frame={tb},
	breaklines=true,
	columns=[l]{fullflexible},
	numbers=left,
	xrightmargin=0zw,
	xleftmargin=3zw,
	numberstyle={\scriptsize},
	stepnumber=1,
	numbersep=1zw,
	lineskip=-0.5ex,
	tabsize=2
}
\renewcommand{\lstlistingname}{ソースコード}

\title{VHDLによるディジタル回路の設計(自由課題)}
\adviser{木場 隼介 先生}

\sdate{令和5年10月19日}
\edate{令和5年11月2日}
\fdate{令和5年11月7日}
\rdate{}

\grade{5}
\anumber{}
\gnumber{B}
\name{河合 将暉}
\jname{}
\comment{}
\begin{document}
\maketitle

\section{目的}
	自由課題を通してVHDLの構文規則や処理方法について詳しく知るとともに、
	プレゼンテーションを行い、課題に対する説明および発表能力を養うことを目的とする。
\section{自由課題}
	\subsection{仕様}
		本実験の自由課題のコンセプトとして、2人で対戦できるルーレットをFPGAで構成することを
		目標とした。仕様としては、FPGAデバッグボードに搭載されている10個のLEDのうち、
		左右から3個ずつのLEDを用いてルーレットを2組構成した。
		ルーレットのストップ・リセットにはFPGAボードに標準搭載されているタクトスイッチ4個を
		使用して1人あたりストップ・リセット用に2個スイッチを割り当てた。

		対戦のルールとして、ルーレットが揃う(LED3個が同色になる)と1点加点され、
		先に2点獲得したプレイヤーの勝利というようなルールを提案する。
		設計はこのルールをベースに設計を行ったが、ルール変更による拡張性にも視野に含めて
		設計した。
	\subsection{参考にしたプログラム}
		自由課題の対戦型ルーレットシステムを構成する際に\cite{ref:指導書}のSample08.vhdを参考に作成した。
		Sample08ではデバックボード上のLEDを3個、FPGAボード上のタクトスイッチを2個用いてルーレットを構成されていた。
		また、Sample08ではルーレット機能のみを記述したroulette.vhdをポートとして呼び出し、マッピングを行ってルーレット機能を
		動作させていた。

	\subsection{使用器具}
		以下に、本課題で使用した器具を表\refeq{tab:used}に示す。
	\begin{table}[H]
	\begin{center}
	\caption{使用器具}
	\label{tab:used}
	\begin{tabular}{clllll} \toprule
	No&\multicolumn{1}{l}{機器名}&\multicolumn{1}{l}{型番}&\multicolumn{1}{l}{シリアルNo}&\multicolumn{1}{l}{備考}\\ \hline
	1&FPGAボード&Cyclone V E FPGA&2&シリアルNoは\\
	&&Development Kit&&外箱の番号を記載\\
	2&PC&ASUS TAF-Gaming&&\\
	\bottomrule
	\end{tabular}
	\end{center}
	\end{table}

\clearpage
\section{プログラム解説}
	\subsection{ピン割当}
		本課題でのデバッグボード上LEDのピン割当を表\refeq{tab:LEDpin}に示す。
		\begin{table}[H]
		\begin{center}
		\caption{デバッグボードLEDのピン割当}
		\label{tab:LEDpin}
		\begin{tabular}{cc|c} \toprule
			ピン名称&入出力&ピン番号\\ \hline
			led\_out0[0]&出力&PIN\_AF21\\
			led\_out0[1]&出力&PIN\_AJ20\\
			led\_out0[2]&出力&PIN\_AG22\\
			led\_out0[3]&出力&PIN\_AK20\\
			led\_out1[0]&出力&PIN\_AF20\\
			led\_out1[1]&出力&PIN\_AJ19\\
			led\_out1[2]&出力&PIN\_AG21\\
			led\_out1[3]&出力&PIN\_AK18\\
			led\_out2[0]&出力&PIN\_AF18\\
			led\_out2[1]&出力&PIN\_AJ17\\
			led\_out2[2]&出力&PIN\_AF19\\
			led\_out2[3]&出力&PIN\_AJ18\\
			led\_out3[0]&出力&PIN\_AG18\\
			led\_out3[1]&出力&PIN\_AG24\\
			led\_out3[2]&出力&PIN\_AG19\\
			led\_out3[3]&出力&PIN\_AH25\\
			led\_out4[0]&出力&PIN\_AK16\\
			led\_out4[1]&出力&PIN\_AH19\\
			led\_out4[2]&出力&PIN\_AK17\\
			led\_out4[3]&出力&PIN\_AH20\\
			led\_out5[0]&出力&PIN\_AF16\\
			led\_out5[1]&出力&PIN\_AG17\\
			led\_out5[2]&出力&PIN\_AG16\\
			led\_out5[3]&出力&PIN\_AH17\\
			led\_out6[0]&出力&PIN\_AE16\\
			led\_out6[1]&出力&PIN\_AJ15\\
			led\_out6[2]&出力&PIN\_AF15\\
			led\_out6[3]&出力&PIN\_AK15\\
			led\_out7[0]&出力&PIN\_AD17\\
			led\_out7[1]&出力&PIN\_AH14\\
			led\_out7[2]&出力&PIN\_AE17\\
			led\_out7[3]&出力&PIN\_AH15\\
			led\_out8[0]&出力&PIN\_AD18\\
			led\_out8[1]&出力&PIN\_AE15\\
			led\_out8[2]&出力&PIN\_AE18\\
			led\_out8[3]&出力&PIN\_AF14\\
			led\_out9[0]&出力&PIN\_Y15\\
			led\_out9[1]&出力&PIN\_AG23\\
			led\_out9[2]&出力&PIN\_AA15\\
			led\_out9[3]&出力&PIN\_AH22\\ \hline
		\end{tabular}
		\end{center}
		\end{table}

	本課題でのFPGAボード上スイッチのピン割当を表\refeq{tab:SWpin}に示す。
		\begin{table}[H]
		\begin{center}
		\caption{FPGAボードのピン割当}
		\label{tab:SWpin}
		\begin{tabular}{c|cc|c} \toprule
			部品名&ピン名称&入出力&ピン番号\\ \hline
			スイッチ&sw\_in1&入力&PIN\_AB12\\
			&sw\_in2&入力&PIN\_AG12\\
			&resetn1&入力&PIN\_AB13\\
			&resetn2&入力&PIN\_AF13\\ \hline
			クロック発振器&clk&入力&PIN\_P22\\
		\bottomrule
		\end{tabular}
		\end{center}
		\end{table}
	\subsection{実装した機能}
		\subsubsection{ルーレットの独立化}


	\subsection{実装できなかった機能}
		\subsubsection{得点表示機能}
		\subsubsection{改善方法の検討}
\section{質疑回答}
	\begin{itemize}
		\item 3人対戦は可能か\\
			現在構成しているシステムではFPGAボードに搭載されているスイッチの数が足りないため、不可能である。\\
		\item LEDを3個しか使用していない理由\\
			FPGAボード上で2人のプレイヤーがスイッチを押そうとすると相手の手で遮られて3個以上は見えづらいため
			ルーレットとしてある程度難しい3個で妥協している。\\
		\item プレイヤー表示を下のLEDの色分けで実装できないか\\
			FPGAボード上のLEDは単色LEDで緑色しか表示できないため、その方式では実装できない。
			デバックボード上ではLEDが4個余っているため、そのLEDを用いれば可能である。\\
		\item リセットボタンを用意するのではなくストップボタンの4回目でリセットにすればよいのではないか\\
			対戦人数を増やすという観点ではいい提案だと感じた。
			今回の課題の設計思想では、2人対戦をメインとしており、ユーザビリティの観点で
			ルーレットが揃わなかった際にボタンを数回連打するのと1回別のボタンを押すのでは
			別のボタンを押したほうがすぐにルーレットをやり直すことができてユーザビリティが
			高いと感じたため、ストップとリセットのスイッチを分けて実装した。\\
	\end{itemize}
\section{所感}
\begin{thebibliography}{99}
\bibitem{ref:指導書}
「実験実習指導書『各種計算ハードウェアの活用〜VHDLによるディジタル回路の設計〜』」\\
神戸高専電子工学科 pp.28-36
\end{thebibliography}
\end{document}